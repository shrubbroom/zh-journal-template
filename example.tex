% !TeX encoding = UTF-8
% !TeX program = xelatex
% !TeX spellcheck = en_US

\documentclass{cjc}

\usepackage{booktabs}
\usepackage{algorithm}
\usepackage{algorithmic}
\usepackage{siunitx}
\usepackage{lipsum}

\classsetup{
  % 配置里面不要出现空行
  title        = {lipsum},
  title*       = {Title},
  journal      = {Lorem Ipsum},
  authors      = {
    author1 = {
      name         = {lipsum},
      name*        = {lipsum},
      affiliations = {lipsum},
      biography    = {},
      % 英文作者介绍内容包括:出生年, 学位(或目前学历), 职称, 主要研究领域(与中文作者介绍中的研究方向一致).
      biography*   = {},
      email        = {lipsum},
      %phone-number = {},  % 第1作者手机号码(投稿时必须提供,以便紧急联系,发表时会删除)
    },
  },
  % 论文定稿后,作者署名、单位无特殊情况不能变更。若变更,须提交签章申请,
  % 国家名为中国可以不写,省会城市不写省的名称,其他国家必须写国家名。
  affiliations = {
    aff1 = {
      name  = {lipsum,lipsum,lipsum},
      name* = {lipsum},
    },
  },
  abstract     = {
    \lipsum[5]
  },
  abstract*    = {},
  % 中文关键字与英文关键字对应且一致,应有5-7个关键词,不要用英文缩写
  keywords     = {lipsum},
  keywords*    = {},
  grants       = {
    % 本课题得到……基金中文完整名称(No.项目号)、
    % ……基金中文完整名称(No.项目号)、
    % ……基金中文完整名称(No.项目号)资助.
  },
  % clc           = {TP393},
  % doi           = {10.11897/SP.J.1016.2020.00001},  % 投稿时不提供DOI号
  % received-date = {2019-08-10},  % 收稿日期
  % revised-date  = {2019-10-19},  % 最终修改稿收到日期,投稿时不填写此项
  % publish-date  = {2020-03-16},  % 出版日期
  % page          = 512,
}

\newcommand\dif{\mathop{}\!\mathrm{d}}

% hyperref 总是在导言区的最后加载
\usepackage{hyperref}



\begin{document}

\maketitle
\lipsum[1]

\lipsum[2]

\lipsum[3]



\begin{acknowledgments}

\end{acknowledgments}


\nocite{*}

\bibliographystyle{cjc}
\bibliography{example}


\end{document}

% \begin{figure}[htb]
%   \centering
%   \includegraphics[width=\linewidth]{example-fig.pdf}
%   \caption{图片说明 *字体为小 5 号,图片应为黑白图,图中的子图要有子图说明*}
% \end{figure}

% \begin{table}[htb]
%   \centering
%   \caption{表说明}
%   \small
%   \begin{tabular}{cc}
%     \toprule
%     示例表格 & 第一行为表头,表头要有内容 \\
%     \midrule
%              & \\
%     \midrule
%              & \\
%     \bottomrule
%   \end{tabular}
% \end{table}

%%% Local Variables:
%%% mode: latex
%%% TeX-master: t
%%% End:
